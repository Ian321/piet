\hypertarget{debug_8h}{
\section{Dokumentacja pliku src/debug.h}
\label{debug_8h}\index{src/debug.h@{src/debug.h}}
}
Plik nagłówkowy debuggera.  


\subsection*{Definicje}
\begin{CompactItemize}
\item 
\#define \hyperlink{debug_8h_86ee3ff44c537d94ccbabf941a613688}{debug}(...)
\end{CompactItemize}


\subsection{Opis szczegółowy}
Plik nagłówkowy debuggera. 

Plik zawiera makro definiujące operacje debuggera. 

\subsection{Dokumentacja definicji}
\hypertarget{debug_8h_86ee3ff44c537d94ccbabf941a613688}{
\index{debug.h@{debug.h}!debug@{debug}}
\index{debug@{debug}!debug.h@{debug.h}}
\subsubsection[{debug}]{\setlength{\rightskip}{0pt plus 5cm}\#define debug( {\em ...})}}
\label{debug_8h_86ee3ff44c537d94ccbabf941a613688}


Procedura wykorzystywana w fazie implementowania aplikacji: do kontroli pamięci, kolejności operacji itp. Gdy aplikacja jest funkcjonalna, procedura nie jest już używana. 